Several proposals for generating synthetic XML data have been
proposed~\cite{FK99, Aboulnaga01generating}. In the latest
proposal~\cite{Aboulnaga01generating}, a data generator accepts a
number of parameters to allow a user to control the properties of the
generated data.  The user needs to define at least twenty parameters
to generate a data set.  It is difficult to understand how some of
these parameters affect the over all properties of the data. An
example of these parameters is the fraction of path tree nodes with
repetition in the tag name and no restrictions on internal or leaf.
Furthermore, the schema of generated data is not available for some
systems that may be able to take advantage of it.

Two benchmarks have been proposed for evaluating the performance of
XML data management systems~\cite{xmach,xmark}.  Both benchmarks
generate XML data that models data from particular applications.
In~\cite{xmach}, the data is based on a web application that consists
of text documents, schema-less data, and structured
data. In~\cite{xmark}, the data is based on an Internet auction
application that consists of relatively structured and data-oriented
part.  Both of these benchmarks do not identify characteristics of XML
data that affect the performance of the XML data management systems.
The benchmark queries are defined, but these queries do not reflect
how the performance of the system changes as the XML data characteristics 
are varied.

Most recently, a desiderata for a benchmark for XML databases that
identifies components and operations that the benchmark should
cover~\cite{SIGMODRECORD01Bench}.  Although our benchmark is not a
general purpose benchmark, our benchmark queries cover all of the ten
challenges that cover all performance critical aspects of XML processing.



