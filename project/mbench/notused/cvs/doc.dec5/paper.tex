\documentclass[fleqn,11pt]{article}
\usepackage{sty/fullpage}

% to recognize url address 
\usepackage{url}

% to order references automatically
\usepackage{sty/cite}

% \usepackage{sty/psfig}
% \usepackage{subfigure}
%\usepackage{lib/tex/defpaper2e}

\newcommand{\qed}{\hfill \rule{2mm}{3mm}}
\newcommand{\eat}[1]{}
\newcommand{\lbok}{\linebreak[0]}
\newcommand{\reminder}[1]{ [[[ \marginpar{\mbox{$<==$}} #1 ]]] }
\newcommand{\eatreminders}{\renewcommand{\reminder}[1]{}}
\newcommand{\comment}[1]{ {\footnotesize {#1} } }
\newcommand{\QED}{\qed}
\newcommand{\ra}{\mbox {$\rightarrow$}}
\newtheorem{Theorem}{Theorem}[section]
\newtheorem{Corollary}{Corollary}[section]
\newtheorem{Lemma}{Lemma}[section]
\newtheorem{Example}{Example}[section]
\newenvironment{Proof}{\noindent{\bf Proof:}}{\QED}
\newtheorem{defn}{Definition}
\newtheorem{algo}{Algorithm}
\newtheorem{observation}{Observation}
\newtheorem{problem}{Problem}

%\noBox

\marginparwidth 0pt
\oddsidemargin  0pt
\evensidemargin  0pt
\marginparsep 0pt
\topmargin   0pt
\textwidth   6.5in
\textheight  8.5 in

\newcommand{\cheatbaseline}{1.44}
\renewcommand{\baselinestretch}{\cheatbaseline}

\title{\bf The Michigan Benchmark: Towards XML Query Performance Diagnostics}

\eat{
\author{
{\bf H. V. Jagadish} \\
Univ of Michigan\\
jag@eecs.umich.edu
\and
{\bf Jignesh M. Patel}\\
Univ of Michigan\\
jignesh@eecs.umich.edu
\and
{\bf Kanda Runapongsa}\\
Univ of Michigan\\
krunapon@eecs.umich.edu
}
}

%\date{}

\begin{document}

\maketitle

%\papernumber 215 raised 3.7in

%%%%%%%%%%%%%%%%%%%%%%%%%%%%%%%%%%%%%%%%%%%%%%%%%%%%%%%%%%%%%%%%%%%%%%

\begin{abstract}
{\footnotesize
We propose a {\em micro-benchmark} for XML data management.  Our benchmark is
designed to help engineers design better XML query processing engines.  In
consequence, it is different from application-level benchmarks designed to
help users choose between alternative products.  In particular, we attempt to
capture the rich variety of data structure and distribution possible
in XML, and to isolate their effects, without any attempt to imitate any particular application.}
\end{abstract}

\vspace*{0.25in}

%%%%%%%%%%%%%%%%%%%%%%%%%%%%%%%%%%%%%%%%%%%%%%%%%%%%%%%%%%%%%%%%%%%%%%

\section{Introduction}
\label{sec:intro}
\input intro

\section{Related Work}
\label{sec:rel}
\input relatedWork

\section{Benchmark Data Set}
\input dataSet

\section{Benchmark queries}
\label{sec:queries}
\input queries

\section{Reporting Benchmark Numbers}
Since the goal of this benchmark is to test individual operations that are
likely to be important in XML query evaluation, we do not propose a single
benchmark number that can be computed from the individual benchmark query
execution times. While having a single benchmark number can be very effective
in summarizing an application benchmark and comparing two or more systems,
for a non-application specific benchmark, such as this benchmark, it may be
meaningless.

For the purposes of this benchmark, each of queries reported above must be
run with each of the three scaling factors.

\section{The Benchmark in Action}
Some performance numbers here -- but which ones??

\section{Conclusion}

\bibliographystyle{plain}
\renewcommand{\baselinestretch}{1.1}
\begin{small}
\bibliography{ref}
%\bibliography{lib/bib/moredb,lib/bib/hier,lib/bib/dbimp,lib/bib/text,koudas}
%\begin{thebibliography}{10}

\eat{
\bibitem{PMA}
A.~Apostolico and Z.~Galil.
\newblock Pattern matching algorithms.
\newblock {\em Oxford University Press}, 1997.

\bibitem{Bertino91}
E.~Bertino.
\newblock An indexing technique for object oriented databases.
\newblock {\em ICDE}, pages 14--22, 1991.

\bibitem{bertino}
E.~Bertino and W.~Kim.
\newblock Indexing techniques for queries on nested objects.
\newblock {\em IEEE Transactions on Knowledge and Data Engineering 1(2)}, pages
  196--214, 1989.


\bibitem{bpsm98:xml}
T.~Bray, J.~Paoli, and C.~M. Sperberg-McQueen.
\newblock Extensible markup language ({XML}) 1.0.
\newblock W3C Recommendation. Available at
  http://www.w3.org/TR/1998/REC-xml-19980210, Feb. 1998.
}
\eat{
\bibitem{Jay3}
M.~Carey, J.~Kierman, J.~Shanmugasundaram, E.~Shekita, and S.~Subramanian.
\newblock XPERANTO: Middleware for publishing object relational data as XML
  documents.
\newblock {\em Proceedings of VLDB}, pages 646--648, 2000.

\bibitem{cd+94:shore}
M.~J. Carey, D.~J. DeWitt, M.~J. Franklin, N.~E. Hall, M.~L. McAuliffe, J.~F.
  Naughton, D.~T. Schuh, M.~H. Solomon, C.~K. Tan, O.~G. Tsatalos, S.~J. White,
  and M.~J. Zwilling.
\newblock Shoring up persistent applications.
\newblock In {\em Proceedings of the ACM SIGMOD Conference on Management of
  Data}, pages 383--394, 1994.

\bibitem{crf00:quilt}
D.~D. Chamberlin, J.~Robie, and D.~Florescu.
\newblock Quilt: An {XML} query language for heterogeneous data sources.
\newblock In {\em WebDB (Informal Proceedings)}, pages 53--62, 2000.

\bibitem{dffls98:xmlql}
A.~Deutsch, M.~Fernandez, D.~Florescu, A.~Levy, and D.~Suciu.
\newblock {XML-QL}: A query language for {XML}.
\newblock Submission to the World Wide Web Consortium 19-August-1998. Available
  from http://www.w3.org/TR/NOTE-xml-ql., 1998.

\bibitem{Dewitt91}
D.~DeWitt, J.~Naughton, and D.~Schneider.
\newblock An evaluation of non equijoin algorithms.
\newblock {\em Proceedings of ACM SIGMOD}, pages 443--452, 1991.

\bibitem{Fernandez}
M.~Fernandez and D.~Suciu.
\newblock SilkRoute: Trading between relations and XML.
\newblock {\em WWW9}, 2000.

\bibitem{Moerkotte}
T.~Fiebig and G.~Moerkotte.
\newblock Evaluating queries on structure with access support relations.
\newblock {\em Proceedings of WebDB}, 2000.

\bibitem{fk99:rdbxml}
D.~Florescu and D.~Kossman.
\newblock Storing and querying {XML} data using an {RDBMS}.
\newblock {\em IEEE Data Engineering Bulletin}, 22(3):27--34, 1999.

\bibitem{GraefeQueryEvaluation}
G.~Graefe.
\newblock Query evaluation techniques for large databases.
\newblock {\em ACM Computing Surveys}, 25(2):73--170, 1993.

\bibitem{jkss98:ds}
G.~Jacobson, B.~Krishnamurthy, D.~Srivastava, and D.~Suciu.
\newblock Focusing search in hierarchical structures with directory sets.
\newblock In {\em Proceedings of the Seventh International Conference on
  Information and Knowledge Management (CIKM)}, Washington, DC, Nov. 1998.

\bibitem{jlmsv99:qnd}
H.~V. Jagadish, L.~V.~S. Lakshmanan, T.~Milo, D.~Srivastava, and D.~Vista.
\newblock Querying network directories.
\newblock In {\em Proceedings of the ACM SIGMOD Conference on Management of
  Data}, Philadelphia, PA, June 1999.

\bibitem{kilger}
C.~Kilger and G.~Moerkotte.
\newblock Indexing multiple sets.
\newblock {\em Proceedings of VLDB}, pages 180--191, Sept. 1994.

\bibitem{koudas97}
N.~Koudas and K.~C. Sevcik.
\newblock Size separation spatial join.
\newblock {\em Proceedings of ACM SIGMOD}, pages 324--335, May 1997.

\bibitem{Lo96}
M.-L. Lo and C.~V. Ravishankar.
\newblock Spatial hash-joins.
\newblock {\em Proceedings of ACM SIGMOD}, pages 247--258, June 1996.

\bibitem{McHugh}
J.~McHugh, S.~Abiteboul, R.~Goldman, D.~Quass, and J.~Widom.
\newblock Lore: A database management systems for semistructured data.
\newblock {\em SIGMOD Record 26(3)}, pages 54--66, 1997.


\bibitem{mw99:xml-queryopt}
J.~McHugh and J.~Widom.
\newblock Query optimization for {XML}.
\newblock In {\em Proceedings of the International Conference on Very Large
  Databases}, pages 315--326, 1999.
}
\eat{
\bibitem{tukwila}
U.~of~Washington.
\newblock The {Tukwila} system.
\newblock Available from http://data.cs.washington.edu/integration/tukwila/.

\bibitem{niagara}
U.~of~Wisconsin.
\newblock The {Niagara} system.
\newblock Available from http://www.cs.wisc.edu/niagara/.

\bibitem{Patel96}
J.~M. Patel and D.~J. DeWitt.
\newblock Partition based spatial-merge join.
\newblock {\em Proceedings of ACM SIGMOD}, pages 259--270, June 1996.

\bibitem{Quass96}
D.~Quass, J.~Widom, R.~Goldman, H.~K, Q.~Luo, J.~McHugh, A.~Rajaraman,
  H.~Rivero, S.~. Abiteboul, J.~Ullman, and J.~Wiener.
\newblock Lore: A lightweight object repository for semistructured data.
\newblock {\em Proceedings of ACM SIGMOD}, page 549, 1996.

\bibitem{rls98:xql}
J.~Robie, J.~Lapp, and D.~Schach.
\newblock {XML} query language ({XQL}).
\newblock Available from http://www.w3.org/TandS/QL/QL98/pp/xql.html.

\bibitem{s:kweelt}
A.~Sahuguet.
\newblock Kweelt.
\newblock Available from http://db.cis.upenn.edu/Kweelt/.

\bibitem{sm83:ir}
G.~Salton and M.~J. McGill.
\newblock {\em Introduction to modern information retrieval}.
\newblock McGraw-Hill, New York, 1983.

\bibitem{ssb+2000:xmlpublish}
J.~Shanmugasundaram, E.~J. Shekita, R.~Barr, M.~J. Carey, B.~G.~Lindsay,
  H.~Pirahesh, and B.~Reinwald.
\newblock Efficiently publishing relational data as {XML} documents.
\newblock In {\em Proceedings of the International Conference on Very Large
  Databases}, 2000.

\bibitem{stz+99:rdbxml}
J.~Shanmugasundaram, K.~Tufte, C.~Zhang, G.~He, D.~J. DeWitt, and J.~F.
  Naughton.
\newblock Relational databases for querying {XML} documents: Limitations and
  opportunities.
\newblock In {\em Proceedings of the International Conference on Very Large
  Databases}, 1999.

\bibitem{Shekita90}
E.~Shekita and M.~Carey.
\newblock A performance evaluation of pointer based joins.
\newblock {\em Proceedings of ACM SIGMOD}, pages 300--311, 1990.

\bibitem{sreenath}
B.~Sreenath and S.~Seshadri.
\newblock The hCtree: An efficient index structure for object oriented
  databases.
\newblock {\em Proceedings of VLDB}, pages 203--213, Sept. 1994.

\bibitem{Stein}
J.~Stein and D.~Maier.
\newblock Associative access support in GemStone.
\newblock {\em In K. Dittrich and A. Buchmann, On Object Oriented Database
  systems, Springer,Verlag}, pages 323--339, 1991.

\bibitem{znd+2001:contain}
C.~Zhang, J.~Naughton, D.~Dewitt, Q.~Luo, and G.~Lohman.
\newblock On supporting containment queries in relational database management
  systems.
\newblock In {\em Proceedings of the ACM SIGMOD Conference on Management of
  Data}, 2001.
}

%\end{thebibliography}
\end{small}
\renewcommand{\baselinestretch}{\cheatbaseline}

\end{document}
