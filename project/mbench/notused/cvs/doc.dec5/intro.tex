XML query processing has taken on considerable
importance recently, and several XML databases \cite{tamino,
timber, ibm, microsoft, oracle, excelon, poet} have been
constructed, on a variety of platforms. There has naturally been
an interest benchmarking the performance of these systems, and
some benchmarks have been proposed \cite{xmach, xmark}.  The
stated objective of these benchmarks is to assess the performance
of a given XML database in performing a variety of representative
tasks. Such benchmarks are valuable to potential users of a database
system in providing an indication of the performance the user can
expect on their specific application.  The challenge is to devise
benchmarks that are sufficiently representative of the requirements of
``most'' users.  The TPC series of benchmarks accomplished this, with
more or less success, for relational database systems. However, no
benchmarks have been successful in the realm of ORDBMS and OODBMS, we
believe on account of the very nature of extensibility and user
defined functions leading to great heterogeneity in the nature of
use. It is too soon to say whether any of the current XML benchmarks
will be successful in this respect -- we certainly hope that they
will.  A lucid statement of desiderata for such benchmarks has
recently been published~\cite{SIGMODRECORD01Bench} One thing that
benchmarks of this nature do not typically focus on is the performance
of the basic query evaluation operations such as selections, joins,
aggregations etc.  A ``micro-benchmark'' that highlights the
performance of these basic operations can be very helpful to a
database developer in understanding and evaluating various
alternatives for implementing these basic operations.  What are the
algorithmic design choices in implementing these basic operations?
What are the strengths and weaknesses of specific access methods?
Which areas should the developer focus attention on? What is the basis
to choose between two alternative implementations?  Questions of this
nature are central to well-engineered systems.  Yet these are at too
fine a granularity to be addressed effectively by the global
system-level benchmarks. The Wisconsin benchmark~\cite{wiss} provided
the database community with an invaluable tool in this regard, for
relational systems. Our goal in this paper is to develop an equivalent
tool for XML databases. Following in the footsteps of the Wisconsin
benchmark, we call the benchmark proposed in this paper the Michigan
benchmark.

Another issue in this regard is the nature of the data in the test
database used for the benchmark.  If the data is carefully specified
to represent some ``real application'', it is likely to be quite
uncharacteristic for other applications with different data
distributions.  Thus, holistic benchmarks can succeed only if they are
able to find a real application with data characteristics that are
reasonably representative for a large class of different applications.

Our objective is to create a single heterogeneous data set, along with
a number of queries, each of which exercises carefully chosen portions
of the data set, with specific data characteristics.  Thus, each query
tests performance with respect to a particular class of data.

Heterogeneity does not mean randomness, though.  Our attempt is to
provide the necessary attributes to be able to specify queries that
will access precisely the type of data we wish to test.  Random number
generators should be used as sparingly as possible -- in the Wisconsin
benchmark only to generate two attribute values -- and all other data
parameters derived from these values.

In relational systems, selectivity\footnote{Selectivity is really not
a characteristic of the data alone, being dependent on the predicate
as well.  For our purposes, it is helpful to think of a specified
predicate and then considering the nature of the data w.r.t. it.}
(including join selectivity) and cardinality are the two
characteristics of primary interest.  (There may be a few additional
secondary characteristics of interest, such as clustering, and size of
tuple/attribute).  In XML databases, besides these characteristics,
there are several others, having to do with the structure, such as
tree fanout and tree depth. A major contribution of this paper is the
identification of such characteristics, and the impact of these
characteristics on query performance.