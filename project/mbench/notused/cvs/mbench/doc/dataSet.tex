In this section, we first discuss the data characteristics of XML data
sets that can have a significant impact on the performance of query
operations. Then, we present the benchmark data sets and explain our
rationale for the data characteristic parameters that we use in the
benchmark data. 

\subsection{A Discussion of the Data Characteristics}
\label{sec:data-char}

In a relational situation, once a relational scheme is fixed, the
tuple size is known, and the primary data characteristics are the
selectivity of attributes (important for simple selection operations)
and the join selectivity (important for join operations). In the XML
case, there are several complicating characteristics to consider, and
we address these in turn below.

\eat{
\subsubsection{Attribute versus Sub-element}
In XML, in many cases, the same information can reasonably be
represented as an attribute of a given node, or as a sub-element of
it.  However, in terms of storage and processing, such a choice could
potentially have an impact. The attributes of a node are typically
part of the node and are retrieved when the node is retrieved, whereas
sub-elements are stored separately, necessitating a possible
additional access in some cases.  Selections are allowed both on
attribute values and sub-element (either by the tag-names or element
content), and alternative storage representations may have a
significant performance impact on the selection operation.  To
investigate the differences in accesses to attributes and elements,
the benchmark data has both attributes and elements. The queries on
the benchmark data set (to be discussed in Section~\ref{sec:queries})
test selections on both attributes and elements.  }

\subsubsection{Structural Selection}
Selection in XML is often based on patterns.  Queries should be
constructed to consider multi-node patterns of various sorts, and with
various selectivities.  An issue that arises is that of ``conditional
selectivity.'' Consider a simple two node selection pattern. Given
that one of the nodes has been identified, conditioned upon this fact,
the selectivity of the second node can be very different than its
selectivity in the database as a whole.  (One could, of course, have
similar dependencies between different attributes in a relation,
thereby affecting the selectivity of a multi-attribute predicate.
Matters are more complicated in XML because these different attributes
may not be in the same tuple, but rather in different, but
structurally related, elements.)  These dependencies can be modeled
by having the ``cyclic numbering'' schemes with different moduli, as
in the Wisconsin benchmark.

Two other data structural parameters are important for tree-structured
data, such as XML. These are fanout and depth. Let us look at the
issue of depth first. The depth of the data tree can have a
significant performance impact when we are computing
ancestor-descendant (or containment) relationships.  It is possible
that the containment is very distant (not an immediate child), and it
is also possible to have multiple nodes at different levels satisfying
the ancestor and the descendant predicates.  Similarly the fanout of
the data tree can have an impact on the way in which the DBMS stores
the data in the DBMS, and answers queries that are based on selecting
children in a specific order (for example select the last child).

One potential way of testing these parameters (fanout and depth) is to
generate a number of distinct data sets with different values for each
of these parameters and then run queries against each data set. The
drawback of this approach is that we have to deal with a large number
of data sets which makes the benchmark harder to run and to
understand. Rather, the approach that we take in this proposal is to
create a base benchmark data set of depth 16. Then, using a ``level''
attribute, we can restrict the scope of the query to data sets of
certain depth, thereby, quantifying the impact of the depth of the
data tree.

To study the impact of fanout, we generate the data set in the
following way. \new{The distribution of nodes is shown in}~\ref{fig:fanout}
\old{Each level in the tree (recall there are 16 levels) has
a fanout of 2, except for levels 5, 6, 7, and 8.} \new{There are 16 levels
in the tree, and each level has a fanout of 2, except levels 5, 6, 7, and 8.}
\old{The levels} \new{Levels} 5, 6, and
7 have a fanout of 4, where as level 8 has a fanout of 1/4 (at level 8
every fourth node has a single child). This variation in fanout is
designed to permit queries that focus on this one factor in isolation.
For instance, the number of nodes at levels 7 and 9 is 256. Nodes at
level 7 have a fanout of 4, whereas \old{the} \new{} nodes at level 9 have a fanout
of 2. Queries against these two levels can be used to measure the
impact of fanout.

\begin {figure}
\begin {center}
\renewcommand{\baselinestretch}{1.1}
\begin{tabular}{|r|r|r|r|} \hline
{\bf Level} & {\bf Fanout} & {\bf Nodes} & {\bf \% of Nodes} \\ \hline
1 & 2 & 1 & 0.0 \\ 
2 & 2 & 2 & 0.0 \\ 
3 & 2 & 4 & 0.0 \\ 
4 & 2 & 8 & 0.0 \\ 
5 & 4 & 16 & 0.0 \\ 
6 & 4 & 64 & 0.1 \\ 
7 & 4 & 256 & 0.4 \\ 
8 & 0.25 & 1,024 & 1.5 \\ 
9 & 2 & 256 & 0.4 \\ 
10 & 2 & 512 & 0.8 \\ 
11 & 2 & 1,024 & 1.5 \\ 
12 & 2 & 2,048 & 3.1 \\ 
13 & 2 & 4,096 & 6.1 \\
14 & 2 & 8,192 & 12.3 \\ 
15 & 2 & 16,384 & 24.6 \\ 
16 & -- & 32,768 & 49.2 \\ \hline
\end {tabular}
\caption{\label{table:dist}Distribution of the nodes in the base data
set}
\label{fig:fanout}
\renewcommand{\baselinestretch}{\cheatbaseline}
\end {center}
\end {figure}

\subsubsection{Document size}

An XML document is tree-structured.  If this document is stored in a
file system, XML documents themselves may exist in a tree-structured
directory. The question is what constitutes an XML database?  At one
extreme, it could be a single huge tree.  At the other, it could
comprise many small documents.  \eat{ Some XML database
implementations may be unaffected by document granularity -- for
others, this may be a significant issue. We consider three
possibilities for scaling data sets:
\begin{itemize}
\item
All data is in a single tree.  Scaling increases the number of nodes
in the tree linearly with the scaling factor.
\item
The size of each tree is kept fixed.  The number of documents is
increased linearly with the scaling factor. \reminder{To permit a
sufficiently rich tree structure, the minimum tree size will have to
be kept fairly large -- imposes a lower bound on the scaling factor}
\item
Both the size of each document tree and the number of documents is
increased with the scaling factor.  To keep this scaling consistent
with the previous two cases, the increase in each is only proportional
to the square root of the scaling factor.
\end{itemize}
} Once again, to keep the benchmark simple, we choose the most
aggressive choice -- namely a single large document tree, as the
default data set.  If it is important to understand the effect of
document granularity, one can modify the benchmark data set to treat
\old{each level node as the root} \new{each node at a given level as the root} 
of a distinct document.  One can
compare the performance of queries on this modified data set against
queries on the original data set.  \old{We have chosen} \new{We choose} not to include this
experiment in the benchmark due to our desire for simplicity, and our
belief that the document granularity issue is not a critical one to
study.

\eat {We run a couple of queries after breaking up this large document
into smaller documents at a specified depth in the tree just to
highlight the document boundary effect, if any.}

\subsubsection{Scaling}
A good benchmark should scale so that appropriately scaled versions of
the same benchmark can be run on platforms with widely varying
capabilities.  In the relational model, scaling a benchmark data set
is easy -- we simply increase the number of tuples.  In the XML world,
there are many choices one could make, and these could have an effect
on the structure of the data set.  We would like to isolate the effect
of the number of nodes from effects due to other structural changes,
such as depth, fanout, etc.  To minimize such cross-impacts, we
propose to scale the Michigan benchmark in discrete steps, as
discussed below.

The default data set, called {\bf DSx1}, has ~67K nodes, arranged in a
tree of depth 16 and fanout 2 for all \old{the} \new{} levels except levels 5, 6, 7
and 8, which have fanouts of 4, 4, 4, 1/4 respectively. From this data set we
generate two additional ``scaled-up'' data sets, called {\bf DSx10}
and {\bf DSx100} that \old{contain approximately 10 and 100 times respective to the
number of nodes in the base data set.} \new{have number of nodes approximately 10 and 100 times to one in the base data set respectively.}  We achieve this scaling factor
by varying the fanout of the nodes at levels 5-8. For the data set
{\bf DSx10} levels 5--7 have a fanout of 13, whereas level 8 has a
fanout of 1/13. For the data set {\bf DSx100} levels 5--7 have a
fanout of 38, whereas level 8 has a fanout of 1/38. The total number
of nodes in the data sets {\bf DSx10} and {\bf DSx100} is 727K and
6,800K respectively~\footnote{this translates into a scale factor of 10.9x and
101.9x.}.

In the design of the benchmark data set, we deliberately \old{chose to} \new{} keep
the fanout of the bottom few levels of the tree constant.  This design
implies that the percentage of nodes in the lower levels of the tree
(levels 9--16) is nearly constant across all the data sets. This
allows us to easily express queries that focus on a specified
percentage of the total number of nodes in the database. For example,
to select approximately 1/16. of all the nodes, irrespective of the
scale factor, we use the predicate {\sf aLevel = 13}.

\subsection{Schema of Benchmark Data}
\label{sec:bench-data}
The construction of the benchmark data is centered around the element
type {\sf BaseType}. Each {\sf BaseType} element has the following
attributes:
\begin {enumerate}
\item {\sf aUnique1}: A unique integer generated by traversing
the entire data tree in a breadth-first manner.  This attribute also
serves as the element identifier.
\item {\sf aUnique2}: A unique integer generated randomly.
\item {\sf aLevel}: An integer set to store the level of the node.
\item {\sf aFour}: An integer set to {\sf aUnique2} mod 4
\item {\sf aSixteen}: An integer set to {\sf aUnique1 + aUnique2} mod 16
\item {\sf aSixtyFour}: An integer set to {\sf aUnique2} mod 64
\item {\sf aString}: A string value approximately 32 bytes \old{in length} \new{}.
\end {enumerate}

The content of each {\sf BaseType} element is a \old{large} \new{long} string that is
approximately 512 bytes \old{in length} \new{}. The generation of the element
content and the string attribute {\sf aString} is described in
Section~\ref{sec:strings}.

In addition to the attributes listed above, each {\sf BaseType}
element has two sets of subelements. The first subelement is of type
{\sf BaseType}. The number of repetitions of this element is used to
control the fanout, as described in Fig.~\ref{fig:fanout}.  The second
subelement is of type {\sf OccasionalType}, and can occur either 0 or 1
times.  The presence or absence of the {\sf OccasionalType} element is
determined by the value of the attribute {\sf aSixtyFour}.  A {\sf
BaseType} element has a nested (leaf) element of type {\sf
OccasionalType} if the {\sf aSixtyFour} attribute has the value 0.
This leaf element has content that is identical to the content of its
parent.  The attribute {\sf aRef} is a reference to the node with id
value equal to the parent's {\sf aUnique1}$-11$. In case such an ID
does not exist (for the top few nodes in the tree), the attribute
points to the root of the tree.

The XML Schema~\cite{xml-schema} specification of the benchmark data
is shown in Figure~\ref{fig:bench-schema}:

\renewcommand{\baselinestretch}{1.1}
\begin {figure}
\begin{small}
\begin{verbatim}
<?xml version="1.0"?>
<xsd:schema xmlns:xsd="http://www.w3.org/2001/XMLSchema"
        targetNamespace="http://www.eecs.umich.edu/db/mbench/bm.xsd"
        xmlns="http://www.eecs.umich.edu/db/mbench/bm.xsd"
        elementFormDefault="qualified">
<xsd:complexType name="BaseType" mixed="true">
  <xsd:sequence>
    <xsd:element name="eNest" type="BaseType" minOccurs="0">
      <xsd:key name="aU1PK">
        <xsd:selector xpath=".//eNest"/>
        <xsd:field xpath="@aUnique1"/>
      </xsd:key>
      <xsd:unique name="aU2">
        <xsd:selector xpath=".//eNest"/>
        <xsd:field xpath="@aUnique2"/>
      </xsd:unique>
    </xsd:element>
    <xsd:element name="eOccasional" type="OccasionalType" minOccurs="0" maxOccurs="1">
      <xsd:keyref name="aU1FK" refer="aU1PK">
        <xsd:selector xpath="../eOccasional"/>
        <xsd:field xpath="@aUnique1Ref"/>
      </xsd:keyref>
    </xsd:element>
  </xsd:sequence>
  <xsd:attributeGroup ref="BaseTypeAttrs"/>
</xsd:complexType>
<xsd:complexType name="OccassionalType">
  <xsd:simpleContent>
    <xsd:extension base="xsd:string">
      <xsd:attribute name="aUnique1Ref" type="xsd:integer" use="required"/>
    </xsd:extension>
  </xsd:simpleContent>
</xsd:complexType>
<xsd:attributeGroup name="BaseTypeAttrs">
  <xsd:attribute name="aUnique1" type="xsd:integer" use="required"/>
  <xsd:attribute name="aUnique2" type="xsd:integer" use="required"/>
  <xsd:attribute name="aLevel" type="xsd:integer" use="required"/>
  <xsd:attribute name="aFour" type="xsd:integer" use="required"/>
  <xsd:attribute name="aSixteen" type="xsd:integer" use="required"/>
  <xsd:attribute name="aSixtyFour" type="xsd:integer" use="required"/>
  <xsd:attribute name="aString" type="xsd:string" use="required"/>
</xsd:attributeGroup>
<xsd:element name="root" type="BaseType"/>
</xsd:schema>
\end{verbatim}
\end{small}
\caption{Benchmark Specification in XML Schema}
\label{fig:bench-schema}
\end {figure}
\renewcommand{\baselinestretch}{\cheatbaseline}

\eat{ To understand the structure of the benchmark data, let us view
each document in the data set as a tree of nodes of type {\sf
BaseType} (ignore the nodes of type {\sf OccasionalType} for now).  A
parent-child relationship in the tree represents a nesting of a
subelement (child) below another subelement (parent). The root of this
tree is at level 0. For this benchmark, the root of the tree has
exactly ten children. Each of these children (at level 1) is the root
of a subtree, and each subtree has the same number of nodes
(approximately a tenth of the number of nodes in the entire benchmark
data). The fanout of the nodes in each of these subtrees is constant
for that subtree. At level 1, we set the attribute {\sf aFanout} to
the the value of the {\sf aTen} attribute for that node. Note that the
{\sf aFanout} for the nodes below level 1 is determined by the value
of the {\sf aFanout} attribute in the ancestor node at level 1.}

\subsection{Generating the String Attributes and Element Content}
\label{sec:strings}
The content of the elements in the benchmark is a long string.  Since
this \old{long} \new{} string is meant to simulate a piece of text in a natural
language, it is not appropriate to generate this string from a uniform
distribution.  Yet, selecting pieces of text from real sources
introduces all manners of difficulty -- how to maintain roughly
constant size for each string, how to avoid idiosyncrasies associated
with the specific source, how to generate more strings as required for
a scaled benchmark, and so on.  Moreover, we would like to have
benchmark results applicable to a wide variety of languages and domain
vocabularies.

To address the above desiderata, we generate these long strings
synthetically, in a carefully stylized manner.  We begin by creating a
pool of $2^{16}-1$ (a little over sixty thousands)
words\footnote{Roughly twice the number of entries in the second
edition of the Oxford English Dictionary. However half the words that
are used in the benchmark are ``derived'' words, produced by appending
``ing'' to the end of a word.} synthetic words as follows. The words
are divided into 16 buckets, with exponentially growing bucket
occupancy.  The first bucket has only one word, the second has two
words, the third has 4, and so on.  Bucket $i$ has $2^{i-1}$ words.
The words are not \old{words from} \new{meaningful in} the English (or any other) language.
Rather, they are synthetically created, with each word containing
information about the bucket from which it is drawn and the word
number in the bucket.  For example, ``15twentynineB14'' indicates that
this is the 1,529th word from the fourteenth bucket.  To keep the size
of the vocabulary at roughly 30,000, words in the last bucket are
derived from words in the other buckets by adding the suffix ``ing''
(to get exactly $2^{15}$ words in the sixteenth bucket, we add the
dummy word ``oneB0ing'').

The value for the long string is generated from \old{the following
template} \new{the template shown in Figure}~\ref{fig:large-str-gen}, where ``PickWord'' is actually a placeholder for words
picked from the word pool described above.  To pick a word for
``PickWord'', a bucket is chosen, with each bucket equally likely, and
then \old{a word is chosen from within the bucket} \new{a word is picked from the chosen bucket}, with each word equally
likely. Thus, we obtain a discrete Zipf distribution of parameter
roughly 1.  We use the Zipf distribution since it seems to reflect
word occurrence probabilities accurately in a wide variety of
situations.

\begin {center}
\renewcommand{\baselinestretch}{1.1}
\begin {figure}
\begin{small}
\begin{verbatim}
                      Sing a song of PickWord,
                      A pocket full of PickWord
                      Four and twenty PickWord
                      All baked in a PickWord.

                      When the PickWord was opened,
                      The PickWord began to sing;
                      Wasn't that a dainty PickWord
                      To set before the PickWord?

                      The King was in his PickWord,
                      Counting out his PickWord;
                      The Queen was in the PickWord
                      Eating bread and PickWord.

                      The maid was in the PickWord
                      Hanging out the PickWord;
                      When down came a PickWord,
                      And snipped off her PickWord!
\end{verbatim}
\end{small}
\caption{Generation of the String Element Content}
\label{fig:large-str-gen}
\end {figure}
\renewcommand{\baselinestretch}{\cheatbaseline}
\end {center}


The string attribute is simply the first line of the large string that
is stored as the element content.
