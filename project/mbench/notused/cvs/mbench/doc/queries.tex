%\subsection{A Discussion of the Query Characteristics}
In creating the data set above, we have taken care to make it possible
to tease apart data with different characteristics, and to issue
queries with well-controlled but vastly differing data access
patterns.  \old{Our belief is that we should} \new{We} do the same with respect to
queries.  Specifically, we are less interested in the composite
performance of queries that are of application-level value.  Rather,
our focus is on evaluating the cost of individual pieces of core query
functionality.  Knowing these costs, we can estimate the cost of any
complex query by just adding up relevant piecewise costs (keeping in
mind the pipelined nature of evaluation, and the changes in sizes of
intermediate results as operators are pipelined).

One clean way to decompose complex queries is by means of an algebra.
%such as TAX\cite{tax}.
While the benchmark is not tied to any particular algebra, we find it
useful to refer to queries as ``selection queries'', ``join queries''
and the like, to clearly indicate the functionality being stressed by
each benchmark query.  A complex query that involves many of these
simple operations can be expected to take time that varies
monotonically with the time required for these simple components.

In the following subsections, we present different type of queries.  
In these queries, the types of the nodes are assumed to be {\sf BaseType} 
(\verb=eNest= nodes) unless specified otherwise.

\eat{ In this algebra, XML data is manipulated using operators that
are very much like relational operators, except that the unit of
manipulation is a tree rather than a tuple.  The heterogeneity of
structure in a tree is handled by means of a pattern tree supplied as
parameter to each TAX operator.  Bindings of pattern tree nodes to
database nodes permits selected database nodes to be named w.r.t. the
known pattern tree structure, rather than with respect to the
non-uniform database tree structure. \reminder{This para. is perhaps
too much about TAX.  Should be softened, and even deleted entirely if
we can otherwise justify considering the simple set of atomic
operations that we would like to}. }

\subsection{Selection Queries}
Whereas relational selection identifies the tuples that satisfy a
given predicate over its attributes, XML selection is considerably
more complex\old{, as we have seen above} \new{}.  One is typically interested in
identifying the portion of the tree that satisfies a specified
predicate.  Furthermore, the predicate itself could involve conditions
on the attributes (or content) of several structurally related
elements. Thus simply understanding the performance of (complex)
selection queries can be extremely valuable in the case of an XML
database.

\subsubsection{Returned Structure}
In a relation, once a tuple is selected, the tuple is returned. In
XML, once an element (node) is selected, one may return not just the
element, but also some structure around the element, most commonly all
or a part of the sub-tree rooted at the element.  How the data is
stored and when the returned result is materialized can make a huge
difference to query performance.

We consider the following cases:
\begin{enumerate}
\item
Return only the element in question.
\item
Return the entire sub-tree rooted at the element.
\item
Return the element and all its immediate children.
\item
Return the element and selected descendants.
\end{enumerate}

To understand the role of returned structure in query performance, we
use the query: Select all elements with {\sf aSixtyFour = 2}. This
query is run four times, corresponding to the four cases above.  For
the fourth case, only the descendants with {\sf aFour = 1} are
returned.  The remaining queries in the benchmark simply return the
selected node (i.e. option 1 above), except when explicitly specified
otherwise.
%In the case of join queries, all nodes that satisfy
%the predicates are returned (i.e. option 1 for all nodes in the join).
\subsubsection{Simple Selection Queries}
\begin {itemize}

\item {\bf Exact Match Attribute Value Selection}
\begin{itemize}
\item
Selection based on the value of a string attribute.
\begin{itemize}
\item {\bf QS1.} High Selectivity: {\sf aString = ``Sing a song of oneB1''}.
Selectivity is approximately 1/16 (6.2\%).
\item {\bf QS2.} Low Selectivity: {\sf aString = ``Sing a song of oneB4''}.
Selectivity is approximately 1/128 (0.8\%).
\end{itemize}
\item
Selection based on the value of an integer attribute.  We reproduce
the same selectivities as in the string attribute case.  
\begin{itemize}
\item {\bf QS3.} The predicate
{\sf aLevel = 13} has selectivity approximately 6.1\%
\item {\bf QS4.} The 
predicate {\sf aLevel = 10} has selectivity approximately 0.8\%.
\end{itemize}
\item
{\bf QS5.} Multi-attribute selection based on the value of two integer
attributes. {\sf aSixteen = 1 and aFour = 1}. Selectivity: $(1/16) \times (1/4) = 1/64$ (1.6\%).
\end{itemize}

\item {\bf Element Name Selection} \\
{\bf QS6.} Selection based on the element
name  {\sf eOccasional}. Selectivity: 1/64 (1.6\%).

\item {\bf Element Content Selection}
\begin{itemize}
\item
{\bf QS7.} Look for nodes that have {\sf ``oneB4''} as substring in element 
content.
The likelihood of a single ``oneB4'' being from bucket 4 is 1/16,
and given bucket 4, of being this particular word is 1/8.  This gives
a product probability of 1/128.  (This string can also arise in bucket
16.  But the probability there is much smaller $2^{-19}$, and hence
can be ignored). There are 16 ``PickWord''s in the content of
each element, so 16 opportunities for this predicate to be satisfied at
each element, giving an overall selectivity of 16/128 (12.5\%).
\item
{\bf QS8.} Look for {\sf OccasionalType} nodes that  ``oneB4'' in the element content. 
Since there is approximately one {\sf OccasionalType} node in every {\sf BaseType} nodes, The selectivity of this query is identical to
1/64 of the selectivity of query QS7.  The overall selectivity is $(16/128) \times (1/64) = 1/512$ (0.2\%).
\end{itemize}

\item {\bf Order-based Selection}
\begin{itemize}
\item
{\bf QS9.} Select the second child of every node with {\sf aLevel = 7}.
Selectivity is 0.4\%.  
\item
{\bf QS10.} Select the second child of every node with {\sf aLevel = 9}.
Selectivity is again 0.4\%.  The difference in performance between
this and the query above is likely to be on account of fanout. The fraction
of nodes in this query is the same as one in QS9.
\end{itemize}
\item {\bf String Distance Selection}
\begin{itemize}
\item
{\bf QS11.} Select all nodes with element content that the distance between keyword {\sf
``oneB2''} and the keyword {\sf
``twenty''} is not more than three. 
There are two occurrences of ``PickWord'' within four words of {\sf ``twenty''} and
14 occurrences that are further away.  The probability of any one
occurrence of ``oneB2'' being selected is 1/32 (3.1\%).
Thus, the overall selectivity approximately $1/32 \times 2 = 1/16$ (6.2\%).
\item
{\bf QS12.} Select all nodes with element content that the distance between keyword {\sf "oneB5"} and the keyword {\sf "twenty"} is not more than three.
The probability of any one occurrence of ``oneB5'' being selected 
 is 1/256 (0.4\%).
Selectivity approximately 1/128 (0.8\%).
\end{itemize}
\end{itemize}


\subsubsection{Structural Selection Queries}
All queries return the root of the selection pattern only, unless
otherwise specified.
\begin{itemize}
\item
{\bf Parent-Child Selection Queries:}
\begin{itemize}
\item {\bf QS13.} Moderate selectivity of both parent and child.  Select nodes
with {\sf aLevel = 13} that have a child with attribute {\sf aSixteen
= 3}.  The first predicate has selectivity of approximately 1/16. The
child predicate also has a selectivity of 1/16.  Since each node at
level 13 has two children, there are two opportunities to satisfy the
child predicate.  Therefore the overall selectivity of this query is
 $(1/16) \times (1/16) \times 2 ~=~ 1/128$ (0.8\%).

\item {\bf QS14.} High selectivity of parent and low selectivity of child.  Select
nodes with {\sf aLevel = 15} that have a child with attribute {\sf
aSixtyFour = 3}.  The first predicate has selectivity of approximately
1/4.  The child predicate has a selectivity of 1/64.  Following the
same argument as above, the selectivity of the query as a whole is
still 1/128 (0.8\%).

\item {\bf QS15.} Low selectivity of parent and high selectivity of child.  Select
nodes with {\sf aLevel = 11} that have a child with attribute {\sf
aFour = 3}.  The first predicate has selectivity of approximately
1/64.  The child predicate has a selectivity of 1/4.  Following the
same argument as above, the selectivity of the query as a whole is
still 1/128 (0.8\%).
\end{itemize}

\item
{\bf Order-sensitive parent-child queries:}
\begin{itemize}

\item {\bf QS16.} Local Ordering.  Select the second element with {\sf aFour = 1}
below {\em each} element with {\sf aFour = 1}. The number of the second element nodes are approximately $\sum_{l=2}^{l=16} (n_l) \times (1/f_{l-1})$ where $n_l$ is the number of nodes at level $l$ and $f_{l-1}$ is the number of fanout at level $l-1$.  This number is about 1/2 of the total number of elements.  Since the selectivity of the element with {\sf aFour =1} is 1/4, the probability that the second element that has {\sf aFour = 1} and that its parent has {\sf aFour = 1} is 1/16.  Thus, the overall selectivity of this query is approximately $(1/2) * (1/16) ~=~ 1/32$ (3.1\%).

\item {\bf QS17.} Global Ordering.  Select the second element with {\sf aFour = 1}
below
{\em any} element with {\sf aSixtyFour = 1}.  This query returns at most one
element, whereas the previous query returns one for each parent.

\item {\bf QS18.} Reverse Ordering.  Select the last element with {\sf aSixteen = 1} below
each element with {\sf aLevel = 13}.  Approximately 1/16 of the nodes are at
level 13, and each has two children.  Almost 1/8 of these will have at least
one child that satisfies the former predicate (from among whom the last one
must be returned in this query).  Thus the overall query selectivity is
approximately 1/128 (0.8\%).

\end{itemize}
\item
{\bf Ancestor-Descendant Selection Queries:}
\begin{itemize}
\item {\bf QS19.} Moderate selectivity of both ancestor and descendant.
Select nodes with {\sf aLevel = 13} that have a descendant with
attribute {\sf aSixteen = 3}.  The first predicate has selectivity
of approximately 1/16. The second (descendant) predicate also has
a selectivity of 1/16.  Since each node at level 13 has 14
descendants, there are 14 opportunities to satisfy the descendant
predicate.
Note that if an ancestor has multiple descendants that satisfy the predicate {\sf aSixteen = 3}, the ancestor is returned only once. Thus, the selectivity is approximately $(1/16) \times (1 - (15/16)^{14})$ = 3.7\%.

\item {\bf QS20.} High selectivity of ancestor and low selectivity of descendant.
Select nodes with {\sf aLevel = 15} that have a descendant with attribute {\sf
aSixtyFour = 3}.  The first predicate has selectivity of approximately 1/4.
The second predicate has a selectivity of 1/64.  The selectivity of the query
as a whole is approximately 1/128.  Note that there are only sixteen levels
in the data tree.  So a node at level 15 only has no descendants other than
its own two children and this query result is identical to the corresponding
parent-child query above.  The overall selectivity is approximately $(1/4 \times (1 - (63/64)^2)$ = 0.8\%.

\item {\bf QS21.} Low selectivity of ancestor and high selectivity of descendant.
Select nodes with {\sf aLevel = 11} that have a descendant with attribute {\sf
aFour = 3}.  The first predicate has selectivity of approximately 1/64.
The second predicate has a selectivity of 1/4 for one descendant.  Since each
level 11 node has 62 descendants, finding such a descendant is virtually
guaranteed, so that the selectivity of the query as a whole is about $(1/64 \times (1 - (3/4)^{62})$ = 1.6\%. 
\end{itemize}

\item {\bf Ancestor Nesting in Ancestor-Descendant Query:}
In the ancestor-descendant queries above, ancestors are never nested below
other ancestors.  To test the performance of queries when ancestors are
recursively nested below other ancestors we have three other
ancestor-descendant queries.  These queries are variants of the queries
above.

\begin{itemize}
\item {\bf QS22.} Moderate selectivity of both ancestor and descendant.
Select nodes with {\sf aSixteen = 3} that have a descendant with attribute {\sf
aSixteen = 3}.  The first predicate has selectivity of approximately 1/16 (6.2\%).
The second (descendant) predicate also has a selectivity of 1/16 (6.2\%).

\item {\bf QS23.} High selectivity of ancestor and low selectivity of descendant.
Select nodes with {\sf aFour = 3} that have a descendant with attribute {\sf
aSixtyFour = 3}.  The first predicate has selectivity of approximately 1/4 (25\%).
The second predicate has a selectivity of 1/64 (1.6\%).

\item {\bf QS24.} Low selectivity of ancestor and high selectivity of descendant.
Select nodes with {\sf aSixtyFour = 3} that have a descendant with attribute
{\sf aFour = 3}.  The first predicate has selectivity of approximately 1/64 (1.6\%).
The second predicate has a selectivity of 1/4 (25\%).
\end{itemize}

The overall selectivities of these queries cannot be the same as
that of the ``equivalent'' unnested queries on two accounts --
first, the same descendants can now have multiple ancestors they
match, and second, the number of candidate descendants is
different (fewer) since the ancestor predicate can be satisfied by
nodes at any level.  These two forces work in opposite directions,
but may not quite cancel each other out in any specific case.  We
\old{have chosen to} \new{} focus on the local predicate selectivities and keep
these the same for all of these queries (as well as for the
parent-child queries considered before).

\item {\bf Return Structure in Structural Pattern Selection Query:}\\
{\bf QS25.} For the last query above, return not just the root of the selection pattern,
but also the descendant node.  Thus the returned structure is a pair of nodes
with an inclusion relationship between them.

\item {\bf Complex Pattern Selection Queries:}
\begin {itemize}
\item {\bf QS26.} One chain query with three parent-child joins with the selectivity pattern: high-low-low-high, to test the choice of join order in evaluating a complex query. To achieve the desired selectivities, we use the following predicates:  {\sf aFour=3}, {\sf aSixteen=3}, {\sf aSixteen=5} and {\sf aLevel=16}.
\item {\bf QS27.} One twig query with two parent child selection, low selectivity of parent {\sf aLevel=11}, high selectivity of one child {\sf aFour=3} and low selectivity of another child {\sf aSixtyFour=3}.
\item {\bf QS28.} One twig query with two parent child selection, high selectivity of
parent {\sf aFour=1}, low selectivity of one child {\sf aLevel=11} and low
selectivity of another child {\sf aSixtyFour=3}.
\item
{\bf QS29-QS31.} Repeat the above three queries using ancestor-descendant in place of
parent-child.
\item
{\bf QS32.} Repeat last query, but use parent-child for one child in the twig query, and use ancestor-descendant below another child.
\end {itemize}
\end {itemize}

\subsection{Value Join Queries}
\begin {itemize}
\item {\bf QJ1.} Low selectivity join: select nodes based on {\sf aSixtyFour =3}
and join with nodes with {\sf aSixtyFour = 4} based on equality of the {\sf
aLevel} attribute.  1/64 of the nodes satisfy either selection condition,
independent of node level.  Since the conditions are independent, the join
selectivity is simply the product $(1/64) \times (1/64) ~=~ 1/4096$ (0.02\%).
\item {\bf QJ2.} High selectivity join: select nodes based on {\sf aFour =2} and
join with nodes with {\sf aFour = 3} based on equality of the {\sf aLevel}
attribute.  1/4 of the nodes satisfy either selection condition,
independent of node level.  The join selectivity is $1/16$ (6.2\%).
\end {itemize}
In computing the value-based joins above, one would naturally expect {\em
both} nodes participating in the join to be returned.  As such, the return
structure expected is a tree per join-pair.  Each tree has a join-node as
root, and two children, one corresponding to each element participating in
the join.

\subsection{Pointer-based Join Query}
\begin {itemize}
\item {\bf QJ3.} Low selectivity join: select all {\sf OccasionalType} nodes that point to a node with {\sf
aSixtyFour =3}.  Selects 1/64 (1.6\%) of all the {\sf OccasionalType} nodes.
\item {\bf QJ4.} High selectivity join: select all {\sf OccasionalType} nodes that point to a node with {\sf aFour
=3}.  Selects 1/4 (25\%) of all the {\sf eOccasional} nodes.
\end{itemize}
Both of these pointer-based joins are semi-join queries.  The returned
elements are only the {\sf eOccasional} nodes, and not the nodes pointed to.

\subsection{Aggregate Queries}

\begin {itemize}
\item {\bf QA1. Value Aggregation.} Over all nodes at level 15, compute the average
value for the {\sf aSixtyFour} attribute.  Recall that about 1/4 of the nodes
are at level 15 and will participate in this aggregation.

\item {\bf QA2. Value Aggregation with Groupby.} Group nodes by level.  Over all
nodes at each level, compute the average value of the {\sf aSixtyFour}
attribute.  The return structure is a tree, with a dummy root and a child for
each group.  Each lead (child) node has one attribute for the level and one
attribute for the average value.

\item {\bf QA3. Structural Aggregation.} Amongst the nodes at level 11,
find the node(s) with the largest fanout.  1/64 of the nodes are at level
11.  Most nodes at this level have exactly two children.  But 1/64 of these
nodes also have a third child, of type {\sf eOccasional}.  These are the
nodes that must be returned.

\item {\bf QA4. Value Aggregate Selection.} Select elements that have at
least two occurrences of the keyword {\sf oneB1} in their content.
There are sixteen chances for this keyword to appear in the content of each
element, and the probability of its being chosen is 1/16.  Solving the
binomial probability distribution that results, there is about a 1/4 chance
of its occurring at least twice.  So the selectivity of this query is 1/4.

\item {\bf QA5. Structural Aggregate Selection.} Select elements that have at
least two children that satisfy {\sf aFour = 1}.
About 50\% of the database nodes are at level 16 and have no children.
Except about 2\% of the remainder, all have exactly two children, and both
must satisfy the predicate for the node to qualify.
The selectivity of the predicate is 1/4.  So the overall
selectivity of this query is obtained by multiplying these probabilities:
$(1/2) \times (1/4) \times (1/4) ~=~ 1/32 (3.1\%)$.

\item {\bf QA6. Structural Exploration.} For each node at level 7, determine
the height of the sub-tree rooted at this node.  The returned structure is a
tree with a dummy root, with a child for each node at level 7.  This child
leaf node has one attribute that references the node at level 7, and another
attribute that records the height of the sub-tree.
There are only 256 nodes at
level 7, irrespective of the database scaling.  Under each of these, the
sub-tree goes to level 16, and so is exactly ten levels high, again
irrespective of scaling.  However, determining this height may require
exploring substantial parts of the database.
\end {itemize}

\subsection{Miscellaneous Queries}
These are additional important queries to cover all performance
critical aspects of XML Processing~\cite{SIGMODRECORD01Bench}.
\begin {itemize}
\item {\bf QM1. Missing Elements.} Find all {\sf BaseType} elements that
there is no {\sf OccasionalType} elements below them.
\item {\bf QM2. Casting.} Retrieve the integer values of {\sf aUnique2}
attribute of elements that have {\sf aLevel = 13} which has selectivity
approximately 6.1\%.
\end{itemize}

\subsection{Updates}
\begin {itemize}
\item {\bf QU1. Point Inserts.} Insert a new node below the node with
{\sf aUnique1 = 10102}.

\item {\bf QU2. Point Deletes.} Delete the node with
{\sf aUnique1 = 10102} and transfer all its children to its parent.

\item {\bf QU3. Bulk Inserts.} Insert a new node below each node with
{\sf aSixtyFour = 1}.  Each new node has attributes identical to its parent,
except for {\sf aUnique1}, which is set to some new large, unique value, not
necessarily contiguous with the values already assigned in the database.

\item {\bf QU4. Bulk Deletes.} Delete all leaf nodes with {\sf aSixteen = 3}.

\item {\bf QU5. Bulk Load.} Load the original data set from a (set of) document(s).

\item {\bf QU6. Bulk Reconstruction.} Return a set of documents, one for each
sub-tree rooted at level 11 and with a child of type {\sf eOccasional}.

\item {\bf QU7. Restructuring.}
For a node $u$ of type {\sf eOccasional}, let $v$ be the parent of $u$, and
$w$ be the parent of $v$ in the database.  For each such node $u$, make $u$ a
direct child of $w$ in the same position as $v$, and place $v$ (along with
the sub-tree rooted at $v$) under $u$.
\end{itemize}

\eat{
\subsection{Other Queries of Interest}
Projection by itself is not a particularly interesting operation from a performance perspective, though sometimes it can be used to improve the performance of a preceding operator, such as selection or join.
}







