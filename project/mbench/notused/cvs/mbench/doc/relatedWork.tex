Several proposals for generating synthetic XML data have been
proposed~\cite{FK99, Aboulnaga01generating}. In the latest
proposal~\cite{Aboulnaga01generating}, a data generator accepts as
many as 20 parameters to allow a user to control the properties of the
generated data. Such large number of parameters adds the level of complexity 
that may interfere the ease of use of a data generator.
Furthermore, this data generator does not make available the schema 
which some systems could exploit.

Two benchmarks have been proposed for evaluating the performance of
XML data management systems~\cite{xmach,xmark}.  Both benchmarks
generate XML data that models data from particular applications.
In~\cite{xmach}, the data is based on a web application that consists
of text documents, schema-less data, and structured
data. In~\cite{xmark}, the data is based on an Internet auction
application that consists of relatively structured and data-oriented
part.  Both of these benchmarks do not identify characteristics of XML
data that affect the performance of the XML data management systems.
The benchmark queries are defined, but these queries do not reflect
how the performance of the system changes as the XML data characteristics 
are varied.

Most recently, a desiderata for a benchmark for XML databases that
identifies components and operations, and ten challenges that the XML benchmark should
address~\cite{SIGMODRECORD01Bench}.  Although our benchmark is not a
general purpose benchmark, it meets all of the those
challenges that test performance critical aspects of XML processing.



